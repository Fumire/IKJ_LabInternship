% !TeX spellcheck = en_US
% !TeX encoding = UTF-8

\documentclass[aps, 10pt, a4paper]{article}

\usepackage{graphics, graphicx, graphics}
\usepackage{fancyvrb, enumerate}
\usepackage{amsmath, amssymb, amscd, amsfonts}
\usepackage{geometry}
\usepackage{multirow}
\usepackage{url}
\usepackage{tikz}
\usepackage{listings, listing}
\usepackage{color}
\usetikzlibrary{shapes, arrows, calc, positioning}

\definecolor{codegreen}{rgb}{0, 0.6, 0}
\definecolor{codegray}{rgb}{0.5, 0.5, 0.5}
\definecolor{codepurple}{rgb}{0.58, 0, 0.82}
\definecolor{backcolour}{rgb}{0.95, 0.95, 0.92}

\lstdefinestyle{mystyle}
{
    backgroundcolor=\color{backcolour},   
    commentstyle=\color{codegreen},
    keywordstyle=\color{magenta},
    numberstyle=\tiny\color{codegray},
    stringstyle=\color{codepurple},
    basicstyle=\footnotesize,
    breakatwhitespace=false,         
    breaklines=true,                 
    captionpos=b,                    
    keepspaces=true,                 
    numbers=left,                    
    numbersep=5pt,                  
    showspaces=false,                
    showstringspaces=false,
    showtabs=false,                  
    tabsize=2,
    frame=single
}
\lstset{style=mystyle}

\tikzstyle{decision} = [diamond, draw, fill=blue!20, text width=4.5em, text badly centered, node distance=3cm, inner sep=0pt]
\tikzstyle{block} = [rectangle, draw, fill=blue!20, text width=5em, text centered, rounded corners, minimum height=2em]
\tikzstyle{line} = [draw, -latex']
\tikzstyle{cloud} = [draw, ellipse, fill=red!20, node distance=5em, minimum height=2em]
\tikzset
{
    -|-/.style=
    {
        to path=
        {
            (\tikztostart) -| ($(\tikztostart)!#1!(\tikztotarget)$) |- (\tikztotarget)
            \tikztonodes
        }
    },
    -|-/.default=0.5,
    |-|/.style=
    {
        to path=
        {
            (\tikztostart) |- ($(\tikztostart)!#1!(\tikztotarget)$) -| (\tikztotarget)
            \tikztonodes
        }
    },
    |-|/.default=0.5,
}

\geometry
{
    top = 20mm,
    bottom = 20mm,
    left = 20mm,
    right = 20mm
}

\title{Lab Internship\\Summer 2019}
\author{Jaewoong Lee}
\date{\today}

\begin{document}
    \maketitle
    \newpage
    
    \tableofcontents
    \listoftables
    \listoffigures
    \listoflistings
    \newpage
    
    \section{Introduction}
        \subsection{Single Cell Analysis}
        
        \subsection{Single Cell Trajectory}
    
    \section{Method}
        \subsection{Cell Ranger}
            Cell Ranger is a set of analysis pipelines that process Chromium single-cell RNA-seq output to align reads, generate feature-barcode matrices and perform clustering and gene expression analysis. \cite{ref:cellranger}
        
        \subsection{Seurat}
            Seurat is an R package designed for QC, analysis, and exploration of single-cell RNA-seq data. Seurat aims to enable users to identify and interpret sources of heterogeneity from single-cell transcriptomic measurements, and to integrate diverse types of single-cell data. \cite{ref:seurat1, ref:seurat2}
        
        \subsection{Scikit-Learn}
            Scikit-Learn is a Python module for machine learning built on top of SciPy. \cite{ref:scikit}
    
    \section{Result}
    
    \section{Discussion}
    
    \section{Acknowledgment}
    
    \addcontentsline{toc}{section}{References}
    \bibliographystyle{apalike}
    \bibliography{reference}

    
\end{document}